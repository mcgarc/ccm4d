In this section we will introduce a basic mathematical description of a diatmoic
molecule, focusing mainly on the simplest case of \ce{H2+}.

Start by considering two hydrogen atoms, one of them ionised, separated by a
distance $R = \infty$. The physics of the lone proton is trivial, and that of
the atom has already been discussed.

If we now decrease $R$ so that the atom and the proton come together, we can
anticipate some electrostatic interaction between the various charges. The
system has a Hamiltonian
%
\begin{equation}
  H = \frac{e^2}{4\pi\epsilon_0 R} - \frac{e^2}{4\pi\epsilon_0 r_1} -
  \frac{e^2}{4\pi\epsilon_0 r_2} +H_\text{kinetic}
\end{equation}
%
where $\mathbf{r}_i$ is the displacement of the electron from proton $i$ and
$H_\text{kinetic}$ is the energy due to the motion of the particles (which we
will return to shortly).

We are looking for a solution wavefunction which reflects the boundary condition
for $R=\infty$ where the electron will be associated with an atomic
wavefunction. We also have the condition that inversion of the system through
the centre of mass will not change the wavefunction. We therefore propose the
solution
%
\begin{equation}
\psi_\pm(\mathbf{r}_1, \mathbf{r}_2) = \phi(\mathbfr}_1) \pm \phi(\mathbf{r}_2).
\end{equation}
%

