In this section we will introduce a basic mathematical description of a diatmoic
molecule, focusing mainly on the simplest case of \ce{H2+}.

Start by considering two hydrogen atoms, one of them ionised, separated by a
distance $R = \infty$. The physics of the lone proton is trivial, and that of
the atom has already been discussed.

If we now decrease $R$ so that the atom and the proton come together, we can
anticipate some electrostatic interaction between the various charges. The
system has a Hamiltonian
%
\begin{equation}
  H = H_\text{kinetic} + \frac{e^2}{4\pi\epsilon_0 R} -
  \frac{e^2}{4\pi\epsilon_0 r_1} - \frac{e^2}{4\pi\epsilon_0 r_2}
\end{equation}
%
where $\mathbf{r}_i$ is the displacement of the electron from proton $i$ and
$H_\text{kinetic}$ is the energy due to the motion of the particles (which we
will return to shortly).

We are looking for a solution wavefunction which reflects the boundary condition
for $R=\infty$ where the electron will be associated with an atomic
wavefunction. We also have the condition that inversion of the system through
the centre of mass will not change the wavefunction. We therefore propose the
solution
%
\begin{equation}
  \psi_\pm(\mathbf{r}_1, \mathbf{r}_2) = \phi(\mathbf{r}_1) \pm \phi(\mathbf{r}_2).
\end{equation}
%
\ph{Energy calculation}

\ph{Bonding/ anti-bonding, gerade and ungerade for $\pm$}

\subsection{Kinetic terms}

Recall the kinetic term of the Hamiltonian in \ph{ref. eqn.}. This can be
written in the COM frame of the protons as
%
\begin{equation}
  H_\text{kinetic} = - \frac{\hbar^2}{2\mu} \nabla^2_\mathbf{R} % and electron term?
\end{equation}
it is instructive to expand the $\nabla_\mathrm{R}$ operator, as this will
expose the radial and rotational parts of the kinetic Hamiltonian,
\begin{equation}
  H_\text{kinetic} = \frac{\hbar^2}{2\mu R^2}\left[ -\frac{\dd}{\dd R}\left(R^2
  \frac{\dd}{\dd R} \right) + \mathbf{N^2}.
  \label{basics:eq:Hkinetic}
\end{equation}
Here we have introduced $\mathbf{N}$, which is the orbital angular momentum
operator for the nucleii. i.e. the kinetic Hamiltonian has two parts:
oscillatory motion in the radial direction, and rotational (the first and second
terms respectively). The total Hamiltonian is
\begin{equation}
  H = H_\text{electronic} + H_\text{vibrational} + H_\text{rotational}.
\end{equation}
% TODO Do I need to clarify what each term corresponds to?

A Hamiltonian like this could perhaps be solved by a separated wave function
\begin{equation}
  \ket{\psi} = \ket{\psi_e} \ket{\psi_v} \ket{\psi_r}
\end{equation}
however, this is not possible in this case, because the electronic Hamiltonian
term has some dependence on $R$. 

It is at this point that we make the \emph{Born-Oppenheimer approximation}: we
assume that this dependence of the electronic wavefunction on $R$ can be
ignored, so that a separated wavefunction can solve the Schr\"odinger equation
for our Hamiltonian. In making this approximation we are essentialy claiming
that the timescale of nuclear motion and electronic motion are vastly different:
the nucleii are stationary on the timescale of electronic motion (we will
justify this below). This means that $R$ can be treated like a parameter of the
electronic state.
