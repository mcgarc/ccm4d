In this section we will introduce a basic mathematical description of a diatmoic
molecule, focusing mainly on the simplest case of \ce{H2+}.

Start by considering two hydrogen atoms, one of them ionised, separated by a
distance $R = \infty$. The physics of the lone proton is trivial, and that of
the atom has already been discussed.

If we now decrease $R$ so that the atom and the proton come together, we can
anticipate some electrostatic interaction between the various charges. The
system has a Hamiltonian $H = H_n + H_e$, with nuclear and electronic terms
respectively. We begin by considering
%
\begin{equation}
  H_e = \frac{p^2_e}{2m_e} + \frac{e^2}{4\pi\epsilon_0 R} -
  \frac{e^2}{4\pi\epsilon_0 r_1} - \frac{e^2}{4\pi\epsilon_0 r_2}
\end{equation}
%
where $\mathbf{r}_i$ is the displacement of the electron from proton $i$.

We are looking for a solution wavefunction which reflects the boundary condition
for $R=\infty$ where the electron will be associated with an atomic
wavefunction. We also have the condition that inversion of the system through
the centre of mass will not change the wavefunction. We therefore propose the
solution
%
\begin{equation}
  \psi_\pm(\mathbf{r}_1, \mathbf{r}_2) = \phi(\mathbf{r}_1) \pm \phi(\mathbf{r}_2).
\end{equation}
%
\ph{Energy calculation}

\ph{Bonding/ anti-bonding, gerade and ungerade for $\pm$}

\subsection{Kinetic terms}


Recall the nuclear term of the Hamiltonian in \ph{ref. eqn.}. This can be
written in the COM frame of the protons as
%
\begin{equation}
  H_\text{n} = - \frac{\hbar^2}{2\mu} \nabla^2_\mathbf{R} % and electron term?
\end{equation}
it is instructive to expand the $\nabla_\mathrm{R}$ operator, as this will
expose the radial and rotational parts of the kinetic Hamiltonian,
% TODO Check sign
\begin{equation}
  H_n = \frac{\hbar^2}{2\mu R^2}\left[ \frac{\dd}{\dd R}\left(R^2
  \frac{\dd}{\dd R}\right) + \mathbf{N^2} \right].
  \label{basics:eq:Hkinetic}
\end{equation}
%
We can now start to anticipate the solutions: the first term looks like an
oscillation in $R$, and the second term is the usual orbital angular momentum
operator
%
\begin{equation}
  N^2 = \frac{1}{\sin\theta}\frac{\partial}{\partial
  \theta}\left(\sin\theta\frac{\partial}{\partial\theta}\right) +
  \frac{1}{\sin^2\theta}\frac{\partial^2}{\partial^2\phi}.
\end{equation}

At this stage we have a Hamiltonian of the form $H = H_e + H_n$ and
$\ket{\psi_e}$, which is some wave function that solves the Schr\"odinger
equation for $H_e$ only. To find the wavefunctions for $H$, it might make sense
to look for a separable solution $\ket{\psi} = \ket{\psi_e}\ket{\psi_n}$. We
expect $\ket{\psi_n}$ to contain some information about the vibration and
rotation of the nucleii. However, this is not so simple because $H_e$ has
dependence on $R$.

It is at this point that we make the \emph{Born-Oppenheimer approximation}: we
assume that this dependence of the electronic wavefunction on $R$ can be
ignored, so that a separated wavefunction can solve the Schr\"odinger equation
for $H$. In making this approximation we are essentialy claiming that the
timescale of nuclear motion and electronic motion are vastly different: the
nucleii are stationary on the timescale of electronic motion (we will justify
this below). This means that $R$ can be treated like a parameter of the
electronic state.
