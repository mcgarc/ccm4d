\cm{Does this section need a little bit on the more advanced methods for finding
the electron orbitals? e.g. Hartree-Fock (which I think is basically finding
self-consistent solutions with a Slater determinant) or DFT? Any others?}

In this section we will see how the combined vibrational, rotational and
electronic structure can be seen in the spectra of a diatomic molecules. The
description given here will not be a full quantum treatment (see H \& W ch. 17).

\subsection{Electronic structure}

As discussed in section \ph{reference, or fix} the electronic structure of a
diatomic molecule can be thought of in terms of the two constituent atoms being
brought together from infinite separation. If the inter-nuclear separation is
$R$ then clearly when $R \to \infty$ we would expect the electronic structure to
be that of two separate atoms, as described in texts such as
Foot~\cite{Foot2005}. We will now extend this idea to larger molecules.

\subsubsection{Molecular orbital theory}

%TODO I wrote a fair bit of this based on HandW sec. 13 and talking to Freddy,
%needs proper citation and a bit more research.

Molecular orbital (MO) theory is a tool that will give us some intuition as to
the way that higher $n$ and $l$ molecular orbital states are formed from a
linear combination of atomic orbitals. We can represent this concept for \ce{H2}
with an MO (\ph{there is another name for this that I have forgotten...})
diagram as shown in \myfigref{diaspec:fig:HMO}.

\begin{figure}
  % TODO
  \caption{This figure should show the H diagram as can be foudn on the MO
  WIki page.}
  \label{diaspec:fig:HMO}
\end{figure}

Recall that for \ce{H2} in the ground state, the 1s atomic orbital was split into the
even and uneven molecular orbitals
%
\begin{equation}
  \sigma_\text{g,u} 1\mathrm{s} = \frac{1}{\sqrt{2}} ( 1\mathrm{s}_a \pm
  1\mathrm{s}_b ).
\end{equation}
%
For higher orbitals, start by considering \ce{Li2}. We can anticipate that the
molecular ground state will be a combination of the ground state atomic orbitals
just as for \ce{H2}. In the limit that our \ce{Li2} was multiply ionised such
that it has only one electron, we would expect that electron to inhabit the same
$1\sigma$ state as we discussed for \ce{H2}. Indeed, the linear
combination of 1s states allows gives space for four electrons. Where do the
remaining two go?

Just as the 1s states mixed, so too will the 2s states. This yields the
$2\sigma$ and $2\sigma^*$ states depicted in
\myfigref{diaspec:fig:LiMO}. Again, we can expect these to be formed of a linear
combination of the atomic 2s states, in complete analogy with the 1s states.

\begin{figure}
  % TODO
  \caption{This figure should show the Li MO diagram as can be foudn on the MO
  WIki page.}
  \label{diaspec:fig:LiMO}
\end{figure}

The orbitals will be filled according to the same rules as for atoms~\cite{}:
%
\begin{enumerate}
  \item Aufbau (fill lowest orbital first)
  \item Pauli (all sublevels are singly occupied before doubling up with
    electrons of opposite sign spin)
  \item Hund (electron spins align so as to maximise total spin)
\end{enumerate}
%
We can see that in \ce{Li2} that there are four electrons in bonding orbitals
and only two in anti-bonding orbitals. The overall energy contribution of the
bond is negative, so \ce{Li2}  is a stable molecule.

You may wonder why we skipped \ce{He} and went straight into the larger element
\ce{Li}. The MO diagram of \ce{He2} is shown in \myfigref{diaspec:fig:HeMO},
where you will see that the number of bonding and anti-bonding electrons are
equal. Hence the bond does not reduce the energy of the two atoms and \ce{He2}
is unstable. This to be expected, \ce{He} is a noble gas after all.

However, if one of the 1s$\sigma^*$ electrons is excited into 2s$\sigma$ then
the \ce{He2} molecule becomes stable. This is an example of an excimer.

\begin{figure}
  % TODO
  \caption{This figure should show the He MO diagram as in HW fig. 13.3.}
  \label{diaspec:fig:HeMO}
\end{figure}

Higher molecular orbitals will also be formed from a linear combination of
atomic orbitals (LCAO). In other words we write the wave function of an orbital
$\psi_i$ in terms of the overlapping atomic orbitals of similar energy $\phi_i$
%
\begin{equation}
  \psi_i = \sum_j c_{ji} \phi_j
\end{equation}
%
where we need to determine the coefficients $c_{ji}$. This can be achieved by
using the variational principle. We have already assumed one result of LCAO: the
even and odd solutions for \ce{H2+} \ph{ref. eqn.} This result can now be
derived as an example.

% From: https://chem.libretexts.org/Courses/New_York_University/CHEM-UA_127%3A_Advanced_General_Chemistry_I/14%3A_Linear_combination_of_atomic_orbitals

We know that the two atomic orbtials are the 1s states of hydrogen
%
\begin{equation}
  \phi_j(r) = \frac{e^{-\abs{r - r_i}/a_0}}{\sqrt{\pi a_0^3}}
\end{equation}
%
and the electronic part of the Hamiltonian is 
%
\begin{equation}
  H_e = - \frac{e^2}{4\pi\epsilon_0}\left(\frac{1}{\abs{r-r_1}} +
  \frac{1}{\abs{r-r_2}}\right).
\end{equation}
% This isn't really the variational principle, the next bit is...
\ph{Using the variational principle}, the energy of the state $\psi_j$ is
%
\begin{equation}
  E_j = \frac{\int \psi_j^\dagger H_e \psi_j \dd V}{\int \psi_j^\dagger \psi_j.
  \dd V}
\end{equation}
%
We seek to minimise $E_j$ with respect to $c_{ji}$. So re-write as
%
\begin{equation}
  E_j = \frac{H_{11}(c_{j1}^2 + c_{j2}^2) + 2H_{12}c_{j1}c_{j2}}{c_{j1}^2
  + c_{j2}^2 + 2Sc_{j1}c_{j2}}
\end{equation}
%TODO make sure S= < phi_1 | phi_2> is defined above (maybe called J or K?
%
where we define the integral $H_{ij} = \int \phi_i^\dagger H_e \phi_j \dd V$ and
note that $H_{11} = H_{22}$. Minimisation is left as an exercise, the reader
should find that
%
\begin{equation}
  c_{j1} = \pm c_{j2}.
\end{equation}
%
Together with the normalisation condition, we have two possible solutions, so
$\j \in \{ +, - \}$, and
%
\begin{equation}
  \psi_\pm = \frac{1}{\sqrt{2}}(\phi_1 \pm \phi_2)
\end{equation}
%
just as we supposed in \ph{ref equation or section}.

Now the energies can be found by solving the Schrodinger equation, to find that
%
\begin{equation}
  E_\pm(R) = \frac{H_{11}(R) \pm H_{12}(R)}{1 \pm S(R)} + \frac{e^2}{4 \pi
  \epsilon_0 R}.
\end{equation}
%
The resulting orbital energies are plotted in \myfigref{:diaspec:fig:Henergies},
along with results from experiment. Clearly LCAO makes for a decent first
approximation of the energy structure of atomic orbitals, but it is not a
complete description. The complete picture will require us to compute the
self-consistent solution for all interaction between electrons and nucleons (see
chapter four of Foot~\cite{Foot2005}). It is not practical to discuss these more
complete methods in this text and MO theory is enough to give us intution of the
way that molecular orbitals arise from mixing of atomic orbitals. We can
continue to extend this to larger diatomic molecules.

% TODO Is there really no more that I need to say?

\begin{figure}
  % TODO
  \caption{This figure should show the energies $E_\pm(R)$ as a function of $R$,
  as in (for example) the chem libretext source. In caption discuss the bond
  length discrepancy.}
  \label{diaspec:fig:Henergies}
\end{figure}

\begin{figure}
  % TODO
  \caption{This figure should show something like Fig. 11 here:
  https://opentextbc.ca/chemistry/chapter/8-4-molecular-orbital-theory/
  }
  \label{diaspec:fig:2ndperiod}
\end{figure}

Figure \myfigref{diaspec:fig:2ndperiod} shows the MO diagrams for the
homonuclear diatomics formed of elements in the second period. MO theory is not
able to accurately explain the shift in energies, but we assert that these can
be obtained from experiment \cm{cite?}.

% TODO better cite than: https://opentextbc.ca/chemistry/chapter/8-4-molecular-orbital-theory/
We can also explain the changing energetic order of the bonds via \emph{s-p
mixing}. Recall that we assumed that the LCAO mixed atomic orbitals of similar
energies. We therefore expect that if an atom has s and p orbitals of similar
energy, then these will be mixed in the molecular state. This is what happends
for the smaller atoms in the period. Oxygen, fluorine and neon have $\text{2p}$
orbitals with paired electrons. This means these orbitals have higher energy
than the smaller atoms in the period and there is no s-p mixing.

Now consider the electrons filling the molecular orbitals. Again we follow the
rules set out above. The filling for \ce{N2} and \ce{O2} is shown in
\myfigref{diaspec:fig:N2O2}. MO theory is able to predict the paramagnetism of
\ce{O2} due to the unpaired electrons in the $\text{2p}\pi^*$ orbital.

\begin{figure}
  % TODO
  \caption{MO diagrams for N2 and O2, such as HandW 13.4, but note that there is
  something inconsistent about the ordering of the 2p MOs with the above fig
  from opentextbc.ca
  }
  \label{diaspec:fig:2ndperiod}
\end{figure}

% Seems like sigma is an axial overlap of electrons and pi is negative, should
% figure this out and write about it.

% How do point symmetries tie in?

\subsubsection{Extension to heteronuclear molecules}

% Heteronuclear symmetries?

Drawing a MO diagram for a heteronuclear diatomic is slightly more involved than
the homonuclear case due to the lack of symmetry in the atomic energy levels.
For example, take the \ce{CO} molecule, whose MO diagram is shown in
\myfigref{diaspec:fig:CO}. Oxygen has a higher atomic number, and hence the
electrons are more tightly bound to the nucleus, resulting in lower atomic
levels. After the orbitals are drawn, electron filling uses the same rules as
above.

\begin{figure}
  % TODO
  \caption{MO diagram of CO}
  \label{diaspec:fig:CO}
\end{figure}

% TODO Order of the levels?

% TODO Less intutive molecule(s) like HCl (also non-bonding orbtials)
% https://chem.libretexts.org/Courses/Heartland_Community_College/HCC%3A_Chem_161/9%3A_Molecular_Geometry_and_Bond_Theory/9.8%3A_M.O._Theory_and_the_Period_2_Diatomic_Molecules

\subsubsection{Extension to ionic bonds}

\subsubsection{Labelling electron states}

% TODO Need some pictures in here I think

We have already introduced the $\sigma$ and $\pi$ labels for electronic states,
but we have not yet stated their meaning. These states arose from LCAOs, and as
such the labels we chose are a variation on the familiar spdf... labelling
scheme for atomic states. 

The spdf... labels correspond to the atomic orbital angular momentum (OAM)
quantum number $l \in \{0,1,2,3\ldots \}$. However for molecules $l$ is no
longer a good quantum number, as the spherical symmetry is broken by the
presence of an inter-nuclear axis. This symmetry-breaking does not affect the
$m_l$ quantum number, since this is just the projection onto the axis.

Hence the $\sigma\pi\delta\phi$ notation corresponds to a new quantum number
$\lambda = \abs{m_l}$, where the sign of $m_l$ has no significance because the
energy from the orbiting electron does not change with the direction of the
orbit.\footnote{This is different to $m_l$ for an atom undergoing Zeeman
splitting because in the molecular case the splitting of $l$ degeneracy is due
to an electric field, not a magnetic one.} \cm{I feel this footnote is quite an
important concept that I don't really understand and should.}
% Probably related to some misunderstanding about the Stark and Zeeman effects
% in atoms too. Only see quadratic Stark effect in atoms, which has no sign(m_l)
% dependence either (I think). [Tangent, this means we should be able to see
% linear Stark effect in heteronuclears, but not homonuclears, worth revisiting!
% TODO]
%
\ph{We have already seen above that} the electron states for homonuclear diatomics
can be expressed using the quantum numbers $\lambda$, $n$, $l$, where the last
two numbers are those of the atomic states at infintie nuclear distance.

% Quantum states: Lambda and Sigma numbers, capital Sigma, Pi, Lambda states
% (c.f. spdf..)

If we have a molecule with more than one valence electron then the same
projection onto the axis will occur for the total OAM vector of all the
electrons. For this we introduce a new quantum number $\Lambda =
\abs{\sum{\lambda_i}}$, where $\lambda_i$ are the individual electrons' OAM
components in $z$. These can be negative, because it is the total OAM which
determines the energy.

% TODO Exemplify further

Our new quantum number can take values $\Lambda \in \{0,1,2,3,\ldots\}$ and to
go with each of these we have new spectroscopic notation $\Sigma\Pi\Delta\Phi$.

For spin, there is no coupling to the electric field but the symmetry breaking
still takes effect. The quantum number is therefore $\Sigma \in \{-S, 1-S,\ldots
S-1, S\}$ where $S$ is the total electron spin ($2S = \text{no. of electrons}$)
and $\Sigma$ should not be confused with the configuration symbol for $\Lambda =
0$.


% TODO 
We have already seen that the parity is marked by a subscript g or u.

% TODO
In general we write states in the form $^{(2\Sigma + 1)}[\Lambda]_{p}$. Where we
use $[]$ to indicate that $\Lambda$ should be replaced with its equivalent
spectroscopic symbol, and $p \in {g, u}$ is the partiy.
%
\ph{Need some proper examples.}


\subsubsection{Electronic potential of diatomic molecules}

% TODO reference correct section
\ph{Above} we found that the electronic contribution to the ground state energy
can be written as a function of the inter-nuclear separation. However our
calculations did not agree well with experiment (see
\myfigref{diaspec:fig:Henergies}).

However, it can be noted that both the potential we predict with the LCAO and
the experimental result closely resemble potentials of anharmonic quantum
oscillators. This should not be surprising: \cm{we have already introduced the
idea that molecules have vibrational structure}. This comes in the form of the
nucleii oscillating in the potential created by the electrons. \cm{GK said this
in passing, is it definitely right?} We expect the oscillation to be anharmonic
because this will allow disassociation of the molecule if it is sufficiently
excited.

The Born-Oppenheimer approximation allows us to assume that this electronic
potential is unchanged on the timescale of nuclear oscillation. So any
oscillation will occur within some potential that is determined by the quantum
state of the elctrons. We can now consider what these potentials will look like
for different states.

% TODO Need to have discussion of SPDF for mols (Sigma, Pi, etc.) above
%
% Morse
% Leonard-Jones

\subsection{Nuclear motion}

The molecule's nuclei will move around in the potential created by its
electrons. We have made the Born-Oppenheimer approximation, which means that we
have assumed the electrons (and hence their potential) to be stationary on the
timescale of this nuclear motion.

\subsubsection{Vibration}

We know from the preceeding section that the electronic potential is that of an
anharmonic oscillator. Using the Morse potential \ph{ref. eqn.} as our model, we
can write the energy of the oscillator as
%
\begin{equation}
  E_v = \hbar \omega_e (v + \frac{1}{2})[1 - x_e(v+\frac{1}{2})]
\end{equation}
%
\ph{def. symbols in electronics section?} where $v$ is the quantum number of the
vibrational state of the molecule. As $v \rightarrow \infty $ the molecule
reaches the disassociation limit and separates.\footnote{If we had assumed a
perfect harmonic oscillator this disassociation limit would not exist, which
would cause us no end of problems later.}

\subsection{Raman spectroscopy}

Raman scattering describes the process of the interaction of off-resonant light
with a molecule.  Light incident on a diatomic molecule will induce a dipole
moment
%
\begin{equation}
  p(t) = \alpha E_0 \cos(2\pi\nu_p t)
\end{equation}
%
where $\alpha$ is the polarisability of the molecule, and $E_0$ and $\nu_p$ are
the amplitude and frequency of the electric field respectively.

For all diatomic molecules, the polarisability is a function of the
inter-nuclear separation \cite{}, which can be expanded to first order
%
\begin{equation}
\alpha = \alpha(R) = \alpha(R_0) + \frac{\dd \alpha}{\dd R}(R - R_0).
\end{equation}
%
This is where the vibration of the molecule comes into play, because of course
$R$ changes through time as the molecule vibrates. We can write down
%
\begin{equation}
R = R_0 + q \cos (2\pi \nu_\text{vib} t)
\end{equation}
and hence the dipole moment is
\begin{equation}
p(t) = \left[ \alpha(R_0) + \frac{\dd \alpha}{\dd R} q\cos (2\pi\nu_\text{vib}t)
\right] E_0 \cos(2\pi\nu_p t).
\end{equation}
Expanding this out we retrieve
\begin{equation}
p(t) = \alpha(R_0)E_0\cos(2\pi\nu_p t) + \frac{1}{2}\frac{\dd \alpha}{\dd R}
E_0 q \left\{ \cos\left[ 2\pi(\nu_p + \nu_\text{vib})t\right]
+ \cos\left[ 2\pi(\nu_p - \nu_\text{vib})t\right] \right\}.
\end{equation}

Hence from the first order of $\alpha(R)$ we have found that the vibrational
spectrum of a diatomic molecule has a principle line at $\nu_p$ and sidebands at
$\nu_p \pm \nu_\text{vib}$. Clearly expansion to higher orders of $\alpha(R)$
will yield higher-order sidebands. This is the \emph{Raman spectrum} of the
molecule. Negative sidebands are \emph{Stokes lines} and positive are
\emph{anti-Stokes}.
%
The Raman spectrum (including the rotational contribution, which we have not yet
discussed) is shown in Figure \ref{diaspec:fig:totalraman}.

\begin{figure}
  % TODO
  \caption{This figure should show the Raman spectrum, similar to H\&W 12.1}
  \label{diaspec:fig:totalraman}
\end{figure}

Upon scattering the photon, the molecule is excited to a virtual level. When the
excited state decays, it can fall back into the same level (\emph{Rayleigh
scattering}), a higher vibrational state (\emph{Stokes scattering}) or a lower
vibrational level (\emph{anti-Stokes scattering}). These processes are
illustrated in \myfigref{diaspec:fig:ramanscatter}.

\begin{figure}
  % TODO
  \caption{This figure should show the possible Raman scattering processes, some
  hybrid of the Wikipedia fig. and H\&W fig. 12.3.}
  \label{diaspec:fig:ramanscatter}
\end{figure}

The relative intensity of each line in the Stokes spectrum can be found by
considering the relative population of each vibrational level. These will be
populated according to a Boltzman distribution, i.e.
%
\begin{equation}
  \frac{I_\text{anti}}{I_\text{Stokes}} = \frac{n(v=1)}{n(=0)} = e^{-\beta h
  \nu_\text{vib}}.
\end{equation}
%
A typical value of $\nu_\text{vib}$ is \SI{30}{\tera\hertz} and at room
temperature (\SI{300}{\kelvin}) this results in a 0.7\% relative intensity. At
ultracold temperatures (\SI{10}{\micro \kelvin}) the relative intensity is
approximately zero ($e^{- 10^8}$).

\subsubsection{Rotational spectrum}

We will now describe the origin of the scattering lines either side of each
Rayleigh and (anti-)Stokes line in the Raman spectrum, as depicted in
\myfigref{diaspec:fig:totalraman}. 
