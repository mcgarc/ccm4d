In this section we will see how the combined vibrational, rotational and
electronic structure can be seen in the spectra of a diatomic molecules. The
description given here will not be a full quantum treatment (see H \& W ch. 17).

\subsection{Nuclear motion}

\subsubsection{Vibrational spectrum}

Light incident on a diatomic molecule will induce a dipole moment
%
\begin{equation}
p(t) = \alpha E_0 \cos(2\pi\nu_p t)
\end{equation}
%
where $\alpha$ is the polarisability of the molecule, and $E_0$ and $\nu_p$ are
the amplitude and frequency of the electric field respectively.

For all diatomic molecules, the polarisability is a function of the
inter-nuclear separation \cite{}, which can be expanded to first order
%
\begin{equation}
\alpha = \alpha(R) = \alpha(R_0) + \frac{\dd \alpha}{\dd R}(R - R_0).
\end{equation}
%
This is where the vibration of the molecule comes into play, because of course
$R$ changes through time as the molecule vibrates. We can write down
%
\begin{equation}
R = R_0 + q \cos (2\pi \nu_\text{vib} t)
\end{equation}
and hence the dipole moment is
\begin{equation}
p(t) = \left[ \alpha(R_0) + \frac{\dd \alpha}{\dd R} q\cos (2\pi\nu_\text{vib}t)
\right] E_0 \cos(2\pi\nu_p t).
\end{equation}
Expanding this out we retrieve
\begin{equation}
p(t) = \alpha(R_0)E_0\cos(2\pi\nu_p t) + \frac{1}{2}\frac{\dd \alpha}{\dd R}
E_0 q \left\{ \cos\left[ 2\pi(\nu_p + \nu_\text{vib})t\right]
+ \cos\left[ 2\pi(\nu_p - \nu_\text{vib})t\right] \right\}.
\end{equation}

Hence from the first order of $\alpha(R)$ we have found that the vibrational
spectrum of a diatomic molecule has a principle line at $\nu_p$ and sidebands at
$\nu_p \pm \nu_\text{vib}$. Clearly expansion to higher orders of $\alpha(R)$
will yield higher-order sidebands. This is the \emph{Raman spectrum} of the
molecule. Negative sidebands are \emph{Stokes lines} and positive are
\emph{anti-Stokes}.
%
The Raman spectrum (including the rotational contribution, which we have not yet
discussed) is shown in Figure \ref{diaspec:fig:totalraman}.

\begin{figure}
  % TODO
  \caption{This figure should show the Raman spectrum, similar to H\&W 12.1}
  \label{diaspec:fig:totalraman}
\end{figure}

When a photon scatters from a molecule, the molecule is excited to a virtual
level. When the excited state decays, it can fall back into the same level
(\emph{Rayleigh scattering}), a higher vibrational state (\emph{Stokes
scattering}) or a lower vibrational level (\emph{anti-Stokes scattering}). These
processes are illustrated in \myfigref{diaspec:fig:ramanscatter}. This process
is known as \emph{Raman scattering}.

\begin{figure}
  % TODO
  \caption{This figure should show the possible Raman scattering processes, some
  hybrid of the Wikipedia fig. and H\&W fig. 12.3.}
  \label{diaspec:fig:ramanscatter}
\end{figure}

The relative intensity of each line in the Stokes spectrum can be found by
considering the relative population of each vibrational level. These will be
populated according to a Boltzman distribution, i.e.
%
\begin{equation}
  \frac{I_\text{anti}}{I_\text{Stokes}} = \frac{n(v=1)}{n(=0)} = e^{-\beta h
  \nu_\text{vib}}.
\end{equation}
%
A typical value of $\nu_\text{vib}$ is \SI{30}{\tera\hertz} and at room
temperature (\SI{300}{\kelvin}) this results in a 0.7\% relative intensity. At
ultracold temperatures (\SI{10}{\micro \kelvin}) the relative intensity is
approximately zero ($e^{- 10^8}$).

\subsubsection{Rotational spectrum}

We will now describe the origin of the scattering lines either side of the
Rayleigh and (anti-)Stokes lines in the Raman spectrum depicted in
\myref{diaspec:fig:totalraman}.
