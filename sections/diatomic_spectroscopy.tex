In this section we will see how the combined vibrational, rotational and
electronic structure can be seen in the spectra of a diatomic molecules. The
description given here will not be a full quantum treatment (see H \& W ch. 17).

\subsection{Electronic structure}

% TODO Density functional theory (DFT) for compmuting ground states

As discussed in section \ph{reference, or fix} the electronic structure of a
diatomic molecule can be thought of in terms of the two constituent atoms being
brought together from infinite separation. If the inter-nuclear separation is
$R$ then clearly when $R \to \infty$ we would expect the electronic structure to
be that of two separate atoms, as described in texts such as
Foot~\cite{Foot2005}.

\ph{We have already discussed} decreasing $R$ for two \ce{H} atoms in their
ground state, which guides us to the gerade and ungerade electron wavefunctions
for the ground state of a \ce{H2} molecule. We will now further justify this
result and discuss larger molecules.

\subsubsection{Molecular orbital theory}

%TODO I wrote a fair bit of this based on HandW sec. 13 and talking to Freddy,
%needs proper citation and a bit more research.

%Molecular orbital (MO) theory is describes to good (how good?) approxiation the electron
%orbitals for small (how small)
%homonuclear (really? if so why not heteronuclear?) diatmoic molecules.
Recall that for \ce{H2} in the ground state, the 1s atomic orbital was split into the
even and uneven molecular orbitals
%
\begin{equation}
  \sigma_\text{g,u} 1\mathrm{s} = \frac{1}{\sqrt{2}} ( 1\mathrm{s}_a \pm
  1\mathrm{s}_b ).
\end{equation}
%
What will we see happen for higher orbitals? Consider \ce{Li2+}. We know that
for \ce{Li} the atomic structure is $\text{1s}^2\text{2s}$. We can anticipate
that nothing will change for the 1s orbitals, we will still get the even and
uneven solutions as in equation \ph{TODO}. Going one step further, there is
nothing inherently different for the 2s orbitals, so they will also form one
bonding and one anti-bonding orbital. We now have four molecular orbitals, which
we label 2s$\sigma^*$, 2s$\sigma$, 1s$\sigma^*$, and 1s$\sigma$.  The
corresponding molecular orbit diagram is shown in \myfigref{diaspec:fig:LiMO}.

\begin{figure}
  % TODO
  \caption{This figure should show the Li MO diagram as can be foudn on the MO
  WIki page.}
  \label{diaspec:fig:LiMO}
\end{figure}

The orbitals will be filled according to the same rules as for atoms~\cite{}:
%
% TODO expand
\begin{enumerate}
    \item Aufbau
    \item Pauli
    \item Hund
\end{enumerate}
%
We can see that in \ce{Li2+} that there are four electrons in bonding orbitals
and only two in anti-bonding orbitals. The overall energy contribution of the
bond is negative, so \ce{Li2+}  is a stable molecule.

Now briefly consider \ce{He2}, shown in \myfigref{diaspec:fig:HeMO}. Here the
energy contribution to the bond is net positive (there are an equal number of
electrons in bonding and anti-bonding orbitals) so this is an unstable
configureation. This is as we expect, \ce{He} is a noble gas after all. However,
if one of the 1s$\sigma^*$ electrons is excited into 2s$\sigma$ then the
\ce{He2} molecule becomes stable. This is an example of an excimer.

\begin{figure}
  % TODO
  \caption{This figure should show the He MO diagram as in HW fig. 13.3.}
  \label{diaspec:fig:HeMO}
\end{figure}

% TODO Is this really Hartree-Fock? I think it is...

Higher molecular orbitals will also be formed from a linear combination of
atomic orbitals (LCAO). In other words we write the wave function of an orbital
$\psi_i$ in terms of the overlapping atomic orbitals of similar energy $\phi_i$
%
\begin{equation}
  \psi_i = \sum_j c_{ji} \phi_j
\end{equation}
%
where we need to determine the coefficients $c_{ji}$. This can be achieved by
using the variational principle. We have already assumed one result of LCAO: the
even and odd solutions for \ce{H2+} \ph{ref. eqn.} This result can now be
derived as an example.

% From: https://chem.libretexts.org/Courses/New_York_University/CHEM-UA_127%3A_Advanced_General_Chemistry_I/14%3A_Linear_combination_of_atomic_orbitals

We know that the two atomic orbtials are the 1s states of hydrogen
%
\begin{equation}
  \phi_j(r) = \frac{e^{-\abs{r - r_i}/a_0}}{\sqrt{\pi a_0^3}}
\end{equation}
%
and the electronic part of the Hamiltonian is 
%
\begin{equation}
  H_e = - \frac{e^2}{4\pi\epsilon_0}\left(\frac{1}{\abs{r-r_1}} +
  \frac{1}{\abs{r-r_2}}\right).
\end{equation}
% This isn't really the variational principle, the next bit is...
\ph{Using the variational principle}, the energy of the state $\psi_j$ is
%
\begin{equation}
  E_j = \frac{\int \psi_j^\dagger H_e \psi_j \dd V}{\int \psi_j^\dagger \psi_j.
  \dd V}
\end{equation}
%
We seek to minimise $E_j$ with respect to $c_{ji}$. So re-write as
%
\begin{equation}
  E_j = \frac{H_{11}(c_{j1}^2 + c_{j2}^2) + 2H_{12}c_{j1}c_{j2}}{c_{j1}^2
  + c_{j2}^2 + 2Sc_{j1}c_{j2}}
\end{equation}
%TODO make sure S= < phi_1 | phi_2> is defined above (maybe called J or K?
%
where we define the integral $H_{ij} = \int \phi_i^\dagger H_e \phi_j \dd V$ and
note that $H_{11} = H_{22}$. Minimisation is left as an exercise, the reader
should find that
%
\begin{equation}
  c_{j1} = \pm c_{j2}.
\end{equation}
%
Together with the normalisation condition, we have two possible solutions, so
$\j \in \{ +, - \}$, and
%
\begin{equation}
  \psi_\pm = \frac{1}{\sqrt{2}}(\phi_1 \pm \phi_2)
\end{equation}
%
just as we supposed in \ph{ref equation or section}.

Now the energies can be found by solving the Schrodinger equation, to find that
%
\begin{equation}
  E_\pm(R) = \frac{H_{11}(R) \pm H_{12}(R)}{1 \pm S(R)} + \frac{e^2}{4 \pi
  \epsilon_0 R}.
\end{equation}
%
This isn't bad as a first approxiation of the energies...

% Talk about ecperimental discrepancy for H2+ and then
% start to think about segway to next section/ bring in need for experimental
% data. Or post-HF methods
%
% How do point symmetries tie in?
%
% Heteronuclear symmetries?
% 
% N2 and O2, maybe should do something smaller without the swap?
% Should include that MO theory doesn't cover absolutely
% everything. Swapping at N2 cutoff and go over HW fig. 13.5
%
% Prediction of paramegntic O2
%
% Morse
% Leonard-Jones

% Types of bonding (HW sec 13.2)

\subsection{Nuclear motion}

\subsection{Raman spectroscopy}

Raman scattering describes the process of the interaction of off-resonant light
with a molecule.  Light incident on a diatomic molecule will induce a dipole
moment
%
\begin{equation}
  p(t) = \alpha E_0 \cos(2\pi\nu_p t)
\end{equation}
%
where $\alpha$ is the polarisability of the molecule, and $E_0$ and $\nu_p$ are
the amplitude and frequency of the electric field respectively.

For all diatomic molecules, the polarisability is a function of the
inter-nuclear separation \cite{}, which can be expanded to first order
%
\begin{equation}
\alpha = \alpha(R) = \alpha(R_0) + \frac{\dd \alpha}{\dd R}(R - R_0).
\end{equation}
%
This is where the vibration of the molecule comes into play, because of course
$R$ changes through time as the molecule vibrates. We can write down
%
\begin{equation}
R = R_0 + q \cos (2\pi \nu_\text{vib} t)
\end{equation}
and hence the dipole moment is
\begin{equation}
p(t) = \left[ \alpha(R_0) + \frac{\dd \alpha}{\dd R} q\cos (2\pi\nu_\text{vib}t)
\right] E_0 \cos(2\pi\nu_p t).
\end{equation}
Expanding this out we retrieve
\begin{equation}
p(t) = \alpha(R_0)E_0\cos(2\pi\nu_p t) + \frac{1}{2}\frac{\dd \alpha}{\dd R}
E_0 q \left\{ \cos\left[ 2\pi(\nu_p + \nu_\text{vib})t\right]
+ \cos\left[ 2\pi(\nu_p - \nu_\text{vib})t\right] \right\}.
\end{equation}

Hence from the first order of $\alpha(R)$ we have found that the vibrational
spectrum of a diatomic molecule has a principle line at $\nu_p$ and sidebands at
$\nu_p \pm \nu_\text{vib}$. Clearly expansion to higher orders of $\alpha(R)$
will yield higher-order sidebands. This is the \emph{Raman spectrum} of the
molecule. Negative sidebands are \emph{Stokes lines} and positive are
\emph{anti-Stokes}.
%
The Raman spectrum (including the rotational contribution, which we have not yet
discussed) is shown in Figure \ref{diaspec:fig:totalraman}.

\begin{figure}
  % TODO
  \caption{This figure should show the Raman spectrum, similar to H\&W 12.1}
  \label{diaspec:fig:totalraman}
\end{figure}

Upon scattering the photon, the molecule is excited to a virtual level. When the
excited state decays, it can fall back into the same level (\emph{Rayleigh
scattering}), a higher vibrational state (\emph{Stokes scattering}) or a lower
vibrational level (\emph{anti-Stokes scattering}). These processes are
illustrated in \myfigref{diaspec:fig:ramanscatter}.

\begin{figure}
  % TODO
  \caption{This figure should show the possible Raman scattering processes, some
  hybrid of the Wikipedia fig. and H\&W fig. 12.3.}
  \label{diaspec:fig:ramanscatter}
\end{figure}

The relative intensity of each line in the Stokes spectrum can be found by
considering the relative population of each vibrational level. These will be
populated according to a Boltzman distribution, i.e.
%
\begin{equation}
  \frac{I_\text{anti}}{I_\text{Stokes}} = \frac{n(v=1)}{n(=0)} = e^{-\beta h
  \nu_\text{vib}}.
\end{equation}
%
A typical value of $\nu_\text{vib}$ is \SI{30}{\tera\hertz} and at room
temperature (\SI{300}{\kelvin}) this results in a 0.7\% relative intensity. At
ultracold temperatures (\SI{10}{\micro \kelvin}) the relative intensity is
approximately zero ($e^{- 10^8}$).

\subsubsection{Rotational spectrum}

We will now describe the origin of the scattering lines either side of each
Rayleigh and (anti-)Stokes line in the Raman spectrum, as depicted in
\myfigref{diaspec:fig:totalraman}. 
