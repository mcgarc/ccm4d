In this section we will see how the combined vibrational, rotational and
electronic structure can be seen in the spectra of a diatomic molecules. The
description given here will not be a full quantum treatment (see H \& W ch. 17).

\subsection{Nuclear motion}

\c{This should probably go into the "basics" section.}

The molecule's nuclei will move around in the potential created by its
electrons. We have made the Born-Oppenheimer approximation, which means that we
have assumed the electrons (and hence their potential) to be stationary on the
timescale of this nuclear motion.

\subsubsection{Vibration}

We know from the preceeding section that the electronic potential is that of an
anharmonic oscillator. Using the Morse potential \ph{ref. eqn.} as our model, we
can write the energy of the oscillator as
%
\begin{equation}
  E_v = \hbar \omega_e (v + \frac{1}{2})[1 - x_e(v+\frac{1}{2})]
\end{equation}
%
\ph{def. symbols in electronics section?} where $v$ is the quantum number of the
vibrational state of the molecule. As $v \rightarrow \infty $ the molecule
reaches the disassociation limit and separates.\footnote{If we had assumed a
perfect harmonic oscillator this disassociation limit would not exist, which
would cause us no end of problems later.} \cm{Drop this footnote, I think we
should make this approx. and go to SHO}

We can now start to think about the sorts of transitions we can expect to see in
a qualitative way by drawing the wavefunctions of the anharmonic states.
Transitions are more likely \ph{more possible?} when there is greater overlap
between the wavefunctions. This is the Franck-Condon principle, and is outlied
in \myfigref{diaspec:fig:franckcondon}.

\begin{figure}
  % TODO
  \caption{Franck-Condon principle exemplified. See Wikipedia for something like
  what this figure should look like.}
  \label{diaspec:fig:franckcondon}
\end{figure}

\ph{Maths: Hooker section 3.1.1 and 3.1.2}

\subsubsection{Rotation}

As well as vibration, we can expect the molecule to rotate. This leads to an
anergy contribution of 
%
\begin{equation}
  E_J = BhcJ(J+1)
\end{equation}
%
with $J$ as the rotational quantum number and $B$ as the energy of the lowest
rotational state. We can expect $Bhc \ll \hbar\omega_e $, so these rotational
energy levels make a very small shift to the vibrational states.

\subsubsection{Ro-vibrational transitions}

The total energy of a molecular state is
%
\cm{need to make these symbols work with the above, esp. X}
\begin{align}
  E(X, v, J) &= E_\text{vib}(X, v) + E_\text{rot}(J) \\
  &= \hbar\omega_X\left(v + \frac{1}{2}\right) + BhcJ(J+1).
\end{align}
%
% To continue, impose selection rules and then we havve P and R branches. How
% does Q come in? ALso need something about transitions that don't involeve
% change of electrical state, see Hooker sec. 3 (I think).

\subsubsection{Fortrat diagrams}

% TODO

\subsection{Raman spectroscopy}

Raman scattering describes the process of the interaction of off-resonant light
with a molecule.  Light incident on a diatomic molecule will induce a dipole
moment
%
\begin{equation}
  p(t) = \alpha E_0 \cos(2\pi\nu_p t)
\end{equation}
%
where $\alpha$ is the polarisability of the molecule, and $E_0$ and $\nu_p$ are
the amplitude and frequency of the electric field respectively.

For all diatomic molecules, the polarisability is a function of the
inter-nuclear separation \cite{}, which can be expanded to first order
%
\begin{equation}
\alpha = \alpha(R) = \alpha(R_0) + \frac{\dd \alpha}{\dd R}(R - R_0).
\end{equation}
%
This is where the vibration of the molecule comes into play, because of course
$R$ changes through time as the molecule vibrates. We can write down
%
\begin{equation}
R = R_0 + q \cos (2\pi \nu_\text{vib} t)
\end{equation}
and hence the dipole moment is
\begin{equation}
p(t) = \left[ \alpha(R_0) + \frac{\dd \alpha}{\dd R} q\cos (2\pi\nu_\text{vib}t)
\right] E_0 \cos(2\pi\nu_p t).
\end{equation}
Expanding this out we retrieve
\begin{equation}
p(t) = \alpha(R_0)E_0\cos(2\pi\nu_p t) + \frac{1}{2}\frac{\dd \alpha}{\dd R}
E_0 q \left\{ \cos\left[ 2\pi(\nu_p + \nu_\text{vib})t\right]
+ \cos\left[ 2\pi(\nu_p - \nu_\text{vib})t\right] \right\}.
\end{equation}

Hence from the first order of $\alpha(R)$ we have found that the vibrational
spectrum of a diatomic molecule has a principle line at $\nu_p$ and sidebands at
$\nu_p \pm \nu_\text{vib}$. Clearly expansion to higher orders of $\alpha(R)$
will yield higher-order sidebands. This is the \emph{Raman spectrum} of the
molecule. Negative sidebands are \emph{Stokes lines} and positive are
\emph{anti-Stokes}.
%
The Raman spectrum (including the rotational contribution, which we have not yet
discussed) is shown in Figure \ref{diaspec:fig:totalraman}.

\begin{figure}
  % TODO
  \caption{This figure should show the Raman spectrum, similar to H\&W 12.1}
  \label{diaspec:fig:totalraman}
\end{figure}

Upon scattering the photon, the molecule is excited to a virtual level. When the
excited state decays, it can fall back into the same level (\emph{Rayleigh
scattering}), a higher vibrational state (\emph{Stokes scattering}) or a lower
vibrational level (\emph{anti-Stokes scattering}). These processes are
illustrated in \myfigref{diaspec:fig:ramanscatter}.

\begin{figure}
  % TODO
  \caption{This figure should show the possible Raman scattering processes, some
  hybrid of the Wikipedia fig. and H\&W fig. 12.3.}
  \label{diaspec:fig:ramanscatter}
\end{figure}

The relative intensity of each line in the Stokes spectrum can be found by
considering the relative population of each vibrational level. These will be
populated according to a Boltzman distribution, i.e.
%
\begin{equation}
  \frac{I_\text{anti}}{I_\text{Stokes}} = \frac{n(v=1)}{n(=0)} = e^{-\beta h
  \nu_\text{vib}}.
\end{equation}
%
A typical value of $\nu_\text{vib}$ is \SI{30}{\tera\hertz} and at room
temperature (\SI{300}{\kelvin}) this results in a 0.7\% relative intensity. At
ultracold temperatures (\SI{10}{\micro \kelvin}) the relative intensity is
approximately zero ($e^{- 10^8}$).

\subsubsection{Rotational spectrum}

We will now describe the origin of the scattering lines either side of each
Rayleigh and (anti-)Stokes line in the Raman spectrum, as depicted in
\myfigref{diaspec:fig:totalraman}. 
